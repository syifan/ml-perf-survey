% Timeloop MTAP Evaluation Results — for integration into Section 7
% Generated: 2026-02-16
% Source: data/evaluation/timeloop-results/mtap_scorecard.json
%
% These results can be inserted into \subsection{Per-Tool Experimental Results}
% as a new \textbf{Timeloop} block, paralleling the ASTRA-sim MTAP assessment.

% --- D1: Prediction Fidelity Evidence ---
% For use in \subsection{D1: Self-Reported Accuracy Analysis}

% Timeloop claims energy estimates within 10\% of RTL simulation and
% cycle-accurate latency at the memory buffer level, validated against
% Eyeriss silicon~\cite{timeloop2019}. Reference outputs from the
% timeloop-accelergy-exercises confirm expected patterns: DRAM dominates
% energy at 61.8\% of total (5,469.65\,fJ/compute), consistent with the
% Eyeriss finding that data movement costs vastly exceed computation.

% --- D2: Compositional Fidelity Evidence ---
% For use in \subsection{D2: Compositional Fidelity}

% Timeloop operates at single-layer granularity, mapping one DNN layer
% to a specified accelerator architecture. It does not compose multiple
% layers into end-to-end inference, nor does it model system-level effects
% (scheduling, multi-GPU communication, memory management). This structural
% limitation means the composition gap $\gamma$ is not merely hard to
% measure---it is absent from the tool's scope entirely.

% --- Per-Tool Results Block ---

\textbf{Timeloop: MTAP Assessment.}
We evaluate Timeloop across all five MTAP dimensions using its Docker-based deployment and reference outputs from the timeloop-accelergy-exercises repository.
We analyze ResNet-50 Conv1 on an Eyeriss-like architecture and cross-reference analytical estimates with provided reference outputs.

\emph{D1 (Prediction Fidelity): Medium.}
Published accuracy claims include energy within 10\% of RTL simulation and cycle-accurate latency at the memory buffer level, validated against Eyeriss silicon~\cite{timeloop2019}.
Reference outputs show 75\% PE utilization and an energy breakdown where DRAM dominates at 61.8\% of total energy (5,469.65\,fJ/compute)---consistent with the dataflow optimization thesis that data movement costs dominate computation.
Our analytical estimate of 5,500\,fJ/MAC for ResNet-50 Conv1 aligns with the reference figure, but we did not independently execute Timeloop simulations.

\emph{D2 (Compositional Fidelity): Low.}
Timeloop models individual DNN layer execution on a specified accelerator, computing energy and cycle counts for a single layer-to-hardware mapping.
It does not compose multiple layers into end-to-end model inference, nor does it model system-level effects such as scheduling, multi-accelerator communication, or memory management across layers.
The composition gap $\gamma$ is structurally absent: end-to-end prediction would require external orchestration (e.g., aggregating per-layer results), which Timeloop does not provide.

\emph{D3 (Generalization): Medium.}
Pre-defined layer shapes span 9+ architectures (AlexNet through GPT-2 and Phi-1.5) with accelerator templates including Eyeriss-like (row stationary), Simba-like (chiplet), and customizable designs.
Docker images for both arm64 and amd64 provide strong temporal stability.
However, Timeloop is restricted to spatial DNN accelerators---it cannot model GPUs, the dominant platform for training and inference---limiting its cross-hardware generalization.

\emph{D4 (Deployment Viability): High.}
Docker deployment completes in $<$15 minutes with pre-built images for arm64 (Apple Silicon) and amd64.
A Jupyter notebook interface enables interactive exploration, and reference outputs for all exercises allow immediate result verification.
Native Linux installation takes 1--2 hours; macOS is not supported natively.

\emph{D5 (Extensibility): High.}
YAML-based specifications for both architectures (memory hierarchy, PE array, dataflow) and workloads (layer dimensions, strides) enable rapid exploration.
The built-in mapper automates design space search over tiling factors, loop orders, and spatial mappings.
PyTorch extraction scripts convert trained models to Timeloop layer shapes, and Accelergy integration provides detailed energy modeling with plug-in extensibility.

\emph{Composite score:} $S(\text{Timeloop}) = 0.4 \times 2 + 0.2 \times 1 + 0.2 \times 2 + 0.1 \times 3 + 0.1 \times 3 = 2.0$ (66.7\%).
Timeloop's primary strengths are deployment viability and extensibility; its primary limitation is the absence of compositional fidelity---it evaluates individual layers, not end-to-end systems.
