% MICRO 2026 Survey Paper - ML Performance Models
% Template based on IEEE conference format

\documentclass[conference]{IEEEtran}

% Packages
\usepackage{cite}
\usepackage{amsmath,amssymb,amsfonts}
\usepackage{algorithmic}
\usepackage{graphicx}
\usepackage{textcomp}
\usepackage{xcolor}
\usepackage{booktabs}
\usepackage{hyperref}
\usepackage{multirow}

% Custom commands
\newcommand{\todo}[1]{\textcolor{red}{[TODO: #1]}}

\begin{document}

\title{A Survey of Machine Learning Approaches for\\Computer Architecture Performance Modeling}

\author{
\IEEEauthorblockN{Authors TBD}
\IEEEauthorblockA{Affiliations TBD}
}

\maketitle

% ==============================================================================
% ABSTRACT
% ==============================================================================
\begin{abstract}
\todo{Write abstract summarizing the survey scope, methodology, key findings, and contributions. Target: 150-200 words.}
\end{abstract}

\begin{IEEEkeywords}
machine learning, performance modeling, computer architecture, neural networks, survey
\end{IEEEkeywords}

% ==============================================================================
% INTRODUCTION
% ==============================================================================
\section{Introduction}
\label{sec:introduction}

\todo{Motivate the importance of performance modeling in computer architecture.}

\todo{Explain why ML approaches are gaining traction over traditional analytical models.}

\todo{State the scope and contributions of this survey:
\begin{itemize}
    \item Comprehensive taxonomy of ML-based performance modeling techniques
    \item Analysis of X papers from MICRO, ISCA, HPCA, MLSys, etc. (2020-2025)
    \item Comparison of approaches across key dimensions
    \item Identification of open challenges and future directions
\end{itemize}
}

\todo{Outline the paper organization.}

% ==============================================================================
% BACKGROUND
% ==============================================================================
\section{Background}
\label{sec:background}

\subsection{Traditional Performance Modeling}
\label{subsec:traditional-modeling}

\todo{Overview of analytical models, simulation-based approaches, and their limitations.}

\subsection{Machine Learning Fundamentals}
\label{subsec:ml-fundamentals}

\todo{Brief primer on ML techniques relevant to performance modeling: regression, neural networks, graph neural networks, transformers, etc.}

\subsection{Problem Formulation}
\label{subsec:problem-formulation}

\todo{Define the performance modeling problem formally. Inputs, outputs, common metrics (IPC, latency, throughput, power).}

% ==============================================================================
% TAXONOMY
% ==============================================================================
\section{Taxonomy}
\label{sec:taxonomy}

\todo{Present the classification framework for organizing the surveyed papers.}

\subsection{By Modeling Target}
\label{subsec:by-target}

\todo{CPU, GPU, accelerators, memory systems, interconnects, full system.}

\subsection{By ML Technique}
\label{subsec:by-technique}

\todo{Classical ML (linear regression, random forests, etc.), deep learning (MLP, CNN, RNN), graph neural networks, transformers.}

\subsection{By Input Representation}
\label{subsec:by-input}

\todo{Hardware counters, microarchitectural features, program features, workload embeddings.}

% ==============================================================================
% SURVEY OF APPROACHES
% ==============================================================================
\section{Survey of Approaches}
\label{sec:survey}

\subsection{CPU Performance Modeling}
\label{subsec:cpu-modeling}

\todo{Survey papers on ML-based CPU performance prediction.}

\subsection{GPU Performance Modeling}
\label{subsec:gpu-modeling}

\todo{Survey papers on ML-based GPU performance prediction.}

\subsection{Accelerator Performance Modeling}
\label{subsec:accelerator-modeling}

\todo{Survey papers on ML models for DNN accelerators, FPGAs, etc.}

\subsection{Memory System Modeling}
\label{subsec:memory-modeling}

\todo{Survey papers on cache, DRAM, and memory hierarchy modeling.}

\subsection{Cross-Platform and Transfer Learning}
\label{subsec:transfer-learning}

\todo{Survey approaches that generalize across hardware configurations.}

% ==============================================================================
% COMPARISON AND ANALYSIS
% ==============================================================================
\section{Comparison and Analysis}
\label{sec:comparison}

\subsection{Accuracy vs. Training Cost}
\label{subsec:accuracy-cost}

\todo{Compare prediction accuracy against data collection and training overhead.}

\subsection{Generalization Capabilities}
\label{subsec:generalization}

\todo{Analyze how well models generalize to unseen workloads and configurations.}

\subsection{Interpretability}
\label{subsec:interpretability}

\todo{Discuss model interpretability and insights gained from ML models.}

\todo{Create comparison tables summarizing key papers across multiple dimensions.}

% ==============================================================================
% OPEN CHALLENGES
% ==============================================================================
\section{Open Challenges and Future Directions}
\label{sec:challenges}

\subsection{Data Availability and Quality}
\label{subsec:data-challenges}

\todo{Discuss challenges in collecting training data, benchmark diversity.}

\subsection{Model Generalization}
\label{subsec:generalization-challenges}

\todo{Challenges in generalizing to new architectures and workloads.}

\subsection{Integration with Design Flows}
\label{subsec:integration-challenges}

\todo{Challenges in integrating ML models into architecture exploration workflows.}

\subsection{Emerging Opportunities}
\label{subsec:opportunities}

\todo{Foundation models for architecture, hardware-software co-design.}

% ==============================================================================
% CONCLUSION
% ==============================================================================
\section{Conclusion}
\label{sec:conclusion}

\todo{Summarize key findings and takeaways from the survey.}

\todo{Reiterate the most promising directions for future research.}

% ==============================================================================
% REFERENCES
% ==============================================================================
\bibliographystyle{IEEEtran}
\bibliography{references}

\end{document}
